\documentclass[10pt]{article}
%\documentclass[twocolumn]{article}
%\documentclass[onecolumn]{article}
% \usepackage{scrtime} % for \thistime (this package MUST be listed first!)
\DeclareUnicodeCharacter{0301}{\'{e}}
\usepackage{times}
\usepackage{graphicx}
\usepackage{float}
\usepackage[margin=0.75in]{geometry}
\usepackage{fancyhdr}
\usepackage{caption}
\usepackage{notoccite}
\usepackage{pgfplotstable}
\usepackage{soul} %For highlights using \hl
%\usepackage[round]{natbib}
%\setcitestyle{aysep={}} %removes the comma between the author and year in citations
%\usepackage{underscore}
\usepackage{pdfpages}
\usepackage{xcolor,colortbl}%for changing cell colour
\usepackage[normalem]{ulem}
\useunder{\uline}{\ul}{}
\usepackage{xspace}
\usepackage{booktabs}
\usepackage{capt-of}
\pagestyle{fancy}
\setlength{\headheight}{15.2pt}
\setlength{\headsep}{13 pt}
\setlength{\parindent}{28 pt}
\setlength{\parskip}{12 pt}
\pagestyle{fancyplain}
\usepackage[T1]{fontenc}
\usepackage{amsmath}
% \usepackage{color,amsmath,amssymb,amsthm,mathrsfs,amsfonts,dsfont}
\usepackage{xspace}
\usepackage{tikz-cd}
\usepackage{tikz}
\usetikzlibrary{decorations.markings}
\usetikzlibrary{calc, arrows}
\usetikzlibrary{external}
\usepackage{pgfplots}
\pgfplotsset{layers/my layer set/.define layer set={background,main,foreground}{},
        set layers=my layer set,}

\usepackage{listings}
\usepackage{xcolor}
\usepackage{booktabs}
\usepackage{tabulary}

\definecolor{codegreen}{rgb}{0,0.6,0}
\definecolor{codegray}{rgb}{0.5,0.5,0.5}
\definecolor{codepurple}{rgb}{0.58,0,0.82}
\definecolor{backcolour}{rgb}{0.95,0.95,0.92}

%for code
\lstdefinestyle{mystyle}{
	backgroundcolor=\color{backcolour},   
	commentstyle=\color{codegreen},
	keywordstyle=\color{magenta},
	numberstyle=\tiny\color{codegray},
	stringstyle=\color{codepurple},
	basicstyle=\ttfamily\footnotesize,
	breakatwhitespace=false,         
	breaklines=true,                 
	captionpos=b,                    
	keepspaces=true,                 
	numbers=left,                    
	numbersep=5pt,                  
	showspaces=false,                
	showstringspaces=false,
	showtabs=false,                  
	tabsize=2
}

\lstset{style=mystyle}
% \usetikzlibrary{pgfplots.clickable}
% \usepgfplotslibrary{clickable}
% tables
\usepackage{longtable}
\usepackage{booktabs}
\usepackage{multicol}
\usepackage{multirow}
% figs
\captionsetup[figure]{labelfont={color=blue}, font={color=black}}
%\usepackage{subfig}% http://ctan.org/pkg/subfig
\usepackage{subcaption}
%\newsubfloat{figure}% Allow sub-figures
\usepackage{caption}%lable fig caption as fig
%\captionsetup[subfigure]{labelfont=bf, justification=raggedright, labelformat=empty} %no caption label
\usepackage{stackengine} %places caption inside figure?
\captionsetup{subrefformat=empty} %when you reference the subcaption it will be (a) for example %{labelfont={color=blue}}
%\captionsetup[subfigure]{labelsep=colon}


% \usepackage{acronym}
% \usepackage{lineno}%for line numbers
%%%%%%%%%%%%%%%%%%%%%%%%%%%%%%%%%%%%%%%%%%%%%%%%%%%%%%%%%%%%%%%%%%%%%%%%%%%%%%%%
% BIBLIOGRAPHY
%%%%%%%%%%%%%%%%%%%%%%%%%%%%%%%%%%%%%%%%%%%%%%%%%%%%%%%%%%%%%%%%%%%%%%%%%%%%%%%%
\usepackage[backend=biber, giveninits=true, doi=false, isbn=false, natbib=true, url=true, eprint=false, style=authoryear-comp, sorting=nyt, sortcites=ynt, maxcitenames=2, maxbibnames=10, minbibnames = 10, uniquename=false, uniquelist=false, dashed=false]{biblatex} % can change the maxbibnames to cut long author lists to specified length followed by et al., currently set to 99.

%% bibliography for each chapter...
\DeclareFieldFormat[article,inbook,incollection,inproceedings,patent,thesis,unpublished]{title}{#1\isdot} % removes quotes around title
\renewbibmacro*{volume+number+eid}{%
	\printfield{volume}%
	%  \setunit*{\adddot}% DELETED
	\printfield{number}%
	\setunit{\space}%
	\printfield{eid}}
\DeclareFieldFormat[article]{number}{\mkbibparens{#1}}
%\renewcommand*{\newunitpunct}{\space} % remove period after date, but I like it. 
\renewbibmacro{in:}{\ifentrytype{article}{}{\printtext{\bibstring{in}\intitlepunct}}} % this remove the "In: Journal Name" from articles in the bibliography, which happens with the ynt 
\renewbibmacro*{note+pages}{%
	\printfield{note}%
	\setunit{,\space}% could add punctuation here for after volume
	\printfield{pages}%
	\newunit}    
\DefineBibliographyStrings{english}{% clears the pp from pages
	page = {\ifbibliography{}{\adddot}},
	pages = {\ifbibliography{}{\adddot}},
} 
\DeclareFieldFormat{journaltitle}{#1\isdot}
\renewcommand*{\revsdnamepunct}{}%remove comma between last name and first name
\DeclareNameAlias{sortname}{family-given}
% \DeclareNameAlias{sortname}{last-first}
\renewcommand*{\nameyeardelim}{\addspace} % remove comma in text between name and date
\addbibresource{ABC1.bib} % The filename of the bibliography
\usepackage[autostyle=true]{csquotes} % Required to generate language-dependent quotes in the bibliography
\renewrobustcmd*{\bibinitperiod}{}
% you'll have to play with the citation styles to resemble the standard in your field, or just leave them as is here. 
% or, if there is a bst file you like, just get rid of all this biblatex stuff and go back to bibtex. 
%%%%%%%%%%%%%%%%%%%%%%%%%%%%%%%%%%%%%%%%%%%%%%%%%%%%%%%%%%%%%%%%%%%%%%%%%%%%%%%%
%
% generally hyperref needs to be loaded last
\usepackage[hidelinks,colorlinks=true,linkcolor=blue,citecolor=blue,urlcolor=blue]{hyperref}
%\usepackage[hidelinks,colorlinks=false,citecolor=blue,urlcolor=darkbrown]{hyperref}
\tikzexternalize

\lhead{Synthetic Lethality} %This needs to change
\rhead{Alexander Turco}
\title{\sc Building sex-specific synthetic lethality network in cancer - Temp Name}
\author{\sc Alexander Turco}

\providecommand{\figref}[1]{Figure \ref{#1}}  %what?
\providecommand{\tabref}[1]{(Table \ref{#1})}  %what?
\providecommand{\e}[1]{\ensuremath{\times 10^{#1}}}
\newcommand{\seg}{\texttt{Seg}\xspace}
\newcommand{\ecoli}{\mbox{\textit{E.\,coli}}\xspace}
\newcommand{\sclong}{\textit{Saccharomyces cerevisiae}\xspace}
\newcommand{\scshrt}{\mbox{\textit{S.\,cerevisiae}}\xspace}
\newcommand{\sce}{\mbox{\textit{S.\,cerevisiae}}\xspace}
\newcommand{\hslong}{\textit{Homo sapiens}\xspace}
\newcommand{\hsshrt}{\mbox{\textit{H.\,sapiens}}\xspace}
\newcommand{\hse}{\mbox{\textit{H.\,sapiens}}\xspace}
\newcommand{\celong}{\textit{Caenorhabditis elegans}\xspace}
\newcommand{\ceshrt}{\mbox{\textit{C.\,elegans}}\xspace}
\newcommand{\dmlong}{\textit{Drosophila melanogaster}\xspace}
\newcommand{\dmshrt}{\mbox{\textit{D.\,melanogaster}}\xspace}
\newcommand{\atlong}{\textit{Arabidopsis thaliana}\xspace}
\newcommand{\atshrt}{\mbox{\textit{A.\,thaliana}}\xspace}
\newcommand{\pflong}{\textit{Plasmodium falciparum}\xspace}
\newcommand{\pfshrt}{\mbox{\textit{P.\,falciparum}}\xspace}
%%%%BIBLIOGRAPHY

%Supplementary File Table Numbers:
\newcommand{\expdata}{S1\xspace}
\newcommand{\seqdata}{S2\xspace}
\newcommand{\protdata}{S3\xspace}
\newcommand{\blast}{S4\xspace}
\newcommand{\tabvar}{S5\xspace}
%Supplementary File Fig Numbers:

\newcommand{\expcor}{S1\xspace}
\newcommand{\expdistATCC}{S3\xspace}
\newcommand{\specialcell}[2][c]{%
	\begin{tabular}[#1]{@{}c@{}}#2\end{tabular}}
\newcommand{\beginsupplement}{%
	\setcounter{table}{0}
	\renewcommand{\thetable}{S\arabic{table}}%    %thetable references table counter 
	\setcounter{figure}{0}
	\renewcommand{\thefigure}{S\arabic{figure}}%
	\setcounter{equation}{0}
	\renewcommand{\theequation}{S\arabic{equation}}%

}
\renewcommand{\thesection}{}
\renewcommand{\thesubsection}{}
\renewcommand{\thesubsubsection}{}
\usepackage{setspace}
%adjust spacing
\doublespacing
\usepackage{titlesec}

\titlespacing\section{0pt}{12pt plus 2pt minus 2pt}{0pt plus 1pt minus 1pt}
\titlespacing\subsection{0pt}{12pt plus 2pt minus 2pt}{0pt plus 1pt minus 1pt}
\titlespacing\subsubsection{0pt}{12pt plus 2pt minus 2pt}{0pt plus 1pt minus 1pt}

% below 3 lines will put ALL table captions at the top...not sure if
% this is what we want but it is good enough for now

% \usepackage{float}
\floatstyle{plaintop}
\restylefloat{table}
%%%%%%%%%%%%%%%%%%%%%%%%%%%%%%%%%%%%%%%%%%%%%%%%%%%%%%%%%%%%%%%%%%%
        
        \definecolor{atomictangerine}{rgb}{1.0, 0.6, 0.4}
        \definecolor{darkbrown}{rgb}{1.0, 0.56, 0.24}
        \colorlet{darkcol}{black!30!white}
        \colorlet{lightcol}{black!10!white}
        \definecolor{txtcol}{HTML}{F40000}


%%%%%%%%%%%%%%%%%%%%%%%%%%%%%%%%%%%%%%%%%%%%%%%%%%%%%%%%%%%%%%%%%%%
\begin{document}

\widowpenalty10000
\clubpenalty10000

%\linenumbers %for line numbers
\onecolumn
%\twocolumn[  
%       \begin{@twocolumnfalse}
%               \begin{center}
                        \maketitle
%               \end{center}
%                       \bigskip

\thispagestyle{empty}
\noindent \textsuperscript{1} Department of, University, , ON, Canada

\newpage
\tableofcontents %Will add a nice Table of Contents
\newpage
       
\section{Abstract} 

\newpage
\section{Introduction/Background Information}

	%What are they
	\subsection{What are Synthetic Lethal Interactions?}

	%Why are they important
	\subsection{Synthetic Lethal Interactions are Harnessed for Precision Oncology}

	%Building SL networks Pan-cancer
	\subsection{Building Pan-Cancer Synthetic Lethality Networks}

	%Bringing sex differences into the paper
	\subsection{Human Sex Differences add An Additional Layer of Complexity}

	%Bringing everything together
	\subsection{Building Pan-Cancer Synthetic Lethality Networks in a Sex Specific Manner} 

\newpage
\section{Materials and Methods} 
%\label{methods}
%Custom scripts and commands utilized in this analysis can be found on \texttt{GitHub} at \url{https://github.com/opticrom/abcmcmc-thesis4c12}.

%\sloppy For a detailed protocol, see Supplementary files on \texttt{GitHub} at
 %\url{https://github.com/JohannaEnright/LCREntropyProject/}.

	\subsection{TCGA Data}
	
	RNA sequencing (RNA-seq) data was obtained from The Cancer Genome Atlas (TCGA). Raw STAR (Spliced Transcripts Alignment to a Reference) aligned counts for tumor tissue and healthy tissue samples were collected. The Cancer Genome Atlas contains genomic information which spans 33 cancer types. For the purpose of this study, only 12 of the 33 cancer types were considered. We first filtered out sex-biased cancers which include breast invasive carcinoma (BRCA), cervical cell carcinoma (CESC), ovarian serous cystadenocarcinoma (OV), prostate adenocarcinoma (PRAD), testicular germ cell tumors (TGCT), uterine corpus endometrial carcinoma (UCEC), and uterine carcinosarcoma (UCS). The reason for this was due to the fact that we are already aware of sex biases in these cancer types. Next, we filtered out blood cancers as well as cancers which lacked normal tissue gene expression samples. This included adrenocortical carcinoma (ACC), lymphoid neoplasm diffuse large b-cell lymphoma (DLBC), glioblastoma multiforme (GBM), acute myeloid leukemia (LAML), and brain lower grade glioma (LGG). The reason for this was due to the fact that we did not have adequate control samples to compare to tumor samples. Finally, we filtered out cancers that did not have any matching pairs of samples (NT and TP from same individual), as well as cancers that had less than 10 matched sample pairs across both males and females. This included mesothelioma (MESO), skin cutaneous melanoma (SKCM), thymoma (THYM), uveal melanoma (UVM), cholangiocarcinoma (CHOL), pancreatic adenocarcinoma (PAAD), pheochromocytoma and paraganglioma (PCPG), rectum adenocarcinoma (READ), and sarcoma (SARC). We selected matched tumor-normal sample pairs to help control for genetic background and other individual-specific factors that could influence gene expression. Once cancer types were selected, two pan-cancer raw count gene expression matrices were created, one for males and one for females.
	
	\begin{table}[ht]
	\centering
	\caption{List of 12 TCGA Cancer Types With Number of Matched Tumor-Normal Samples in Males and Females.}
	% Second version of table, with booktabs.
	\begin{tabulary}{\textwidth}{CCCC}\toprule
		& \multicolumn{3}{c}{TCGA information}
		\\\cmidrule(lr){2-4}
		& Cancer Type  & Matched Female Samples & Matched Male Samples\\\midrule
		& BLCA & 9 & 10 \\
		& COAD & 21 & 20 \\
		& ESCA & 5 & 8 \\
		& HNSC & 14 & 29 \\
		& KICH & 12 & 13 \\
		& KIRC & 20 & 52 \\
		& KIRP & 10 & 22 \\
		& LIHC & 22 & 28 \\
		& LUAD & 34 & 24 \\
		& LUSC & 14 & 37 \\
		& STAD & 10 & 23 \\
		& THCA & 42 & 17 \\
		& \textcolor{red}{TOTAL} & \textcolor{red}{213} & \textcolor{red}{283} \\\bottomrule
	\end{tabulary}
	\end{table}

	\subsection{Pre-filtering and Normalization of Raw RNA-seq Count Data}
	
	Raw count gene expression matrices for males and females were pre-filtered to remove genes unlikely to exhibit differential expression. For each matrix, we calculated the 90th quantile of overall gene expression as a threshold. For each gene in the matrix, we checked to see whether its expression was greater than the threshold in at least 1 sample. We removed genes where no samples showed an expression value greater than the quantile threshold. 
	
	The pre-filtered matrices were then normalized using the DESeq2 package in R \citep{love2014moderated}. RNA-seq data must be normalized in order to account for factors that prevent the direct comparison of expression measures. The DESeq2 R package employs a median of ratios normalization method to account for the inherent biases associated with RNA-seq data. 

	\subsection{Data Quality Assessment (PCA, NPManova)}
	
	

	\subsection{Differential Gene Expression Analysis with DESeq2}

%	\subsection{One Sample T-Test for Probability of Acceptance}

\section{Results}

%	\subsection{Assessing the Evolution Simulator}

%	\subsection{Applying the Simulator to an ABC-MCMC}

\section{Discussion}

%	\subsection{Mutations Have the Ability to Destroy LCR Formation}

%	\subsection{Does the Length of a Repeat Play a Role in Insertions/Deletions}

%	\subsection{Why Parameters Could not be Estimated}

%	\subsection{Conclusion}

\clearpage\newpage
\section{References}

%%%FIGURES%%%%%

%%%%PRINTING BIBLIOGRAPHY%%%%
\nocite{*}
\printbibliography[heading=none, sorting=nyt]
\newpage

%The following files included these appendices are located at \texttt{/home/alext/scratch/abcmcmc-thesis4c12}

%\section{Appendix 1: Low Complexity Region Evolution Simulator}
%\texttt{/home/alext/scratch/abcmcmc-thesis4c12/mutations\_2\_BG\_vecs.cpp}
%\lstinputlisting[language=c++]{../mutations_2_BG_vecs.cpp}
%\newpage

%\section{Appendix 2: ABC-MCMC in C++}
%\texttt{/home/alext/scratch/abcmcmc-thesis4c12/abcmcmc.cpp}
%\lstinputlisting[language=c++]{../abcmcmc.cpp}
%\newpage

%\section{Appendix 3: Simulated Protein Summary Statistics}
%\texttt{/home/alext/scratch/abcmcmc-thesis4c12/abcmcmc.cpp/simulated\_protein.cpp}
%\lstinputlisting[language=c++]{../simulated_protein.cpp}
%\newpage

%\section{Appendix 4: Euclidean Distance and Normlization of Vectors}
%\texttt{/home/alext/scratch/abcmcmc-thesis4c12/abcmcmc.cpp/distance2.cpp}
%\lstinputlisting[language=c++]{../distance2.cpp}
%\newpage

%\section{Appendix 5: One Sample t-test}
%\texttt{/home/alext/scratch/abcmcmc-thesis4c12/abcmcmc.cpp/ttest.cpp}
%\lstinputlisting[language=c++]{../ttest.cpp}
%\newpage

\end{document}



